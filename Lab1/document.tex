\documentclass[a4paper,12pt]{article}
\usepackage{amsmath,amsthm,amssymb}
\usepackage{mathtext}
\usepackage[top=2cm, bottom=2cm]{geometry}
\usepackage[T1,T2A]{fontenc}
\usepackage[utf8]{inputenc}
\usepackage[russian]{babel}
\usepackage{fancyhdr}
\pagestyle{fancy}
\fancyhf{}
\renewcommand{\headrulewidth}{0pt}
\rhead{\textbf{\thepage}}
\setcounter{page}{315}
\chead{\textbf{Бесконечные непрерывные дроби}}
\lhead{\textbf{Раздел 13-4}}

\begin{document}
	
	\noindent{\textsc{Определение 13-3.} Если $a_0$, $a_1$, $a_2$, ... $\textemdash$ это бесконечная\\
	последовательность целых чисел, которые все, кроме, быть может,  $a_0$,\\
	положительные, то бесконечная простая непрерывная дробь $[a_0; a_1, a_2, ...]$\\
	имеет значение $\lim_{n\to\infty} [a_0; a_1, a_2, ..., a_n]$.}\\
	\\
	\indent{Следует ещё раз подчеркнуть, что прилагательное ``простая''\\
	означает, что все частичные знаменатели $a_k$ являются целыми числами;\\
	поскольку те бесконечные непрерывные дроби, что будут рассмотрены\\
	здесь, являются простыми, мы часто будем опускать этот термин\\
	в дальнейшем и будем называть их бесконечными непрерывными\\
	дробями.}\\\\
	\indent{Пожалуй, самым элементарным примером является бесконечная\\ непрерывная дробь [1; 1, 1, 1, ...]. Пример 13-1 показал, что \textit{n}-я\\
	сходящаяся дробь $C_n$=[1; 1, 1, ..., 1], где целое число 1 встречается n+1 раз, равна\\
	\begin{center}$C_n$=$\frac{u_{n+1}}{u_n}$ (n$\geq0$),\end{center}
	что является частным последовательных чисел Фибоначчи. Если \textit{x} $\textemdash$ это\\
	значение непрерывной дроби [1; 1, 1, 1, ...], тогда}\\
	\\
	 $$\indent{\textit{x}=\lim_{n\to\infty} C_n=\lim_{n\to\infty} \frac{u_{n+1}}{u_n}=\lim_{n\to\infty} \frac{u_n+u_{n-1}}{u_n}=\lim_{n\to\infty} (1+\frac{1}{\frac{u_n}{u_{n-1}}})}=$$\\
	 $$= 1+\frac{1}{\lim_{n\to\infty} (\frac{u_n}{u_{n-1}})}=1+\frac{1}{x}$$\\
 	\\
 	\indent{Это приводит к квадратному уравнению $\textit{$x^2$}-\textit{x}-1=0$, единственный \\
 	положительный корень которого \textit{x}=(1+$\sqrt{5}$)/2. Следовательно,\\
 	\begin{center}$\frac{1+\sqrt{5}}{2}$=[1; 1, 1, 1, ...].\end{center}} \indent{Есть одна ситуация, которая возникает достаточно часто, чтобы\\
 	заслужить особую терминологию. Если бесконечная непрерывная дробь,\\
 	такая как [3; 1, 2, 1, 6, 1, 2, 1, 6, ...] содержит набор частичных\\
 	знаменателей, который бесконечно повторяется, то дробь называется\\
 	\textit{периодической}. Принято записывать периодическую непрерывную дробь\\
 	$[a_0; a_1, ..., a_m, b_1, ..., b_n, b_1, ..., b_n, ...]$ более компактно:\\
 	\begin{center}$[a_0; a_1, ..., a_m, \overline{b_1, ..., b_n}],$\end{center} где линия над $b_1, b_2..., b_n$ означает, что данный набор чисел\\
 	повторяется снова и снова. Если $b_1, b_2..., b_n$ самый маленький блок чисел,\\
 	который постоянно повторяется, то говорят, что $b_1, b_2..., b_n$ $\textemdash$ \textit{период}\\
 	дробного разложения, а его \textit{длина}  равна n. Таким образом, к примеру, $[3; \overline{1, 2, 1, 6}]$ будет означать [3; 1, 2, 1, 6, 1, 2, 1, 6, ...] $\textemdash$ это непрерывная дробь, период которой 1, 2, 1, 6, а его длина равна 4.}
 	\newpage
 	\renewcommand{\headrulewidth}{0pt}
 	\lhead{\textbf{\thepage}}
 	\setcounter{page}{316}
 	\chead{\textbf{Числа Фибоначчи и непрерывные дроби}}
 	\rhead{\textbf{Глава 13}}
 	\indent{Мы видели ранее, что любая конечная непрерывная дробь\\
 	представляется рациональным числом. Теперь давайте рассмотрим значение бесконечной непрерывной дроби.}\\
 	\\
 	\textsc{Теорема 13-9.} \textit{Значение любой бесконечной непрерывной дроби является\\
 	иррациональным числом.}\\\\
	\textit{Доказательство:} Пусть \textit{x} это значение бесконечной\\
	непрерывной дроби $[a_0; a_1, a_2, ...]$; то есть, \textit{x} $\textemdash$ предел последовательности\\
	сходящихся дробей\\
	\begin{center} $C_n$=$[a_0; a_1, a_2, ..., a_n]$=$\frac{p_n}{q_n}$. \end{center}
	Так как \textit{x} находится строго между сходящимися дробями $C_n$ и $C_{n+1}$, мы имеем\\
	\begin{center} 0<|$\textit{x}-C_n$|<|$C_{n+1}-C_n$|=$\left| \frac{p_{n+1}}{q_{n+1}}-\frac{p_n}{q_n} \right|$=$\frac{1}{q_{n}q_{n+1}}$. \end{center}
	Чтобы получить противоречие, предположим, что \textit{x} рациональное число; пусть, например, \textit{x=a/b}, где \textit{a} и \textit{b>0} $\textemdash$ целые числа. Тогда\\
	\begin{center} 0<$\left|\frac{a}{b}-\frac{p_n}{q_n}\right|$<$\frac{1}{q_{n}q_{n+1}}$ \end{center}
	и далее, после умножения на положительное число \textit{b$q_n$}, получается\\
	\begin{center} 0<|\textit{a$q_n-bp_n$}|<$\frac{b}{q_{n+1}}$. \end{center}
	Напомним, что $q_i$ неограниченно возрастает по мере увеличения \textit{i}. Если \textit{n}\\
	выбрано достаточно большим, чтобы выполнялось условие \textit{b}<$q_{n+1}$, то \\
	результатом будет\\
	\begin{center} 0<|\textit{a$q_n-bp_n$}|<1. \end{center}
	Это говорит о том, что существует положительное целое число, а именно\\
	|\textit{a$q_n \textendash bp_n$}|, находящееся между 0 и 1 $\textemdash$ что, очевидно, невозможно.\\\\
	\indent{Теперь зададимся вопросом, могут ли две разные бесконечные\\
	непрерывные дроби представлять одно иррациональное число. Прежде чем дать ответ на этот вопрос, заметим, что свойства пределов позволяют нам записать\\
	бесконечную непрерывную дробь $[a_0; a_1, a_2, ...]$ в виде}\\
	$$[a_0; a_1, a_2, ...]=\lim_{n\to\infty} [a_0; a_1, ..., a_n]=\lim_{n\to\infty} \left(a_0+\frac{1}{[a_1; a_2, ..., a_n]}\right)=$$\\
	$$a_0+\frac{1}{\lim_{n\to\infty} [ a_1; a_2, ..., a_n]}=a_0+\frac{1}{[a_1; a_2, a_3, ...]}.$$
	\newpage
	\renewcommand{\headrulewidth}{0pt}
	\rhead{\textbf{\thepage}}
	\setcounter{page}{317}
	\chead{\textbf{Бесконечные непрерывные дроби}}
	\lhead{\textbf{Раздел 13-4}}
	\indent{Наша теорема формулируется так:}\\\\
	\textsc{Теорема 13-10.} \textit{Если бесконечные непрерывные дроби $[a_0; a_1, a_2, ...]$ и $[b_0; b_1, b_2, ...]$ равны, то $a_n=b_n$ для всех $n\geq0$.}\\\\
	\textit{Доказательство}: Если \textit{x}=$[a_0; a_1, a_2, ...]$, то $C_0$<\textit{x}<$C_1$, а это значит то же самое, что и $a_0<\textit{x}<a_0+1/a_1$. Зная, что $a_1\geq1$, получаем из предыдущего неравенства $a_0<\textit{x}<a_0+1$. Следовательно, [\textit{x}]=$a_0$, где [\textit{x}] $\textemdash$ традиционное обозначение для наибольшего целого числа или ``скобочная'' функция\\
	(стр. 126).\\\\
	\indent{Теперь предположим, что $[a_0; a_1, a_2, ...]$=\textit{x}=$[b_0; b_1, b_2, ...]$, или, если записать это в другом виде,}\\
	$$a_0+\frac{1}{[a_1; a_2, ...]}=\textit{x}=b_0+\frac{1}{[b_1, b_2, ...]}.$$\\
	В силу заключения из первого параграфа, мы имеем, что $a_0=\textit{x}=b_0$, из чего можно сделать вывод, что $[a_1, a_2, ...]=[b_1, b_2, ...].$ Повторяя такое рассуждение, \\
	делаем заключение, что $a_1=b_1$ и $[a_2, a_3, ...]=[b_2, b_3, ...]$. Процесс\\
	продолжается по математической индукции, тем самым давая $a_n=b_n$ для всех $n\geq0$.\\\\
	\textsc{Следствие.} \textit{Две разные бесконечные непрерывные дроби представляют два разных иррациональных числа.}\\\\
	\textbf{Пример 13-4.}\\
	Чтобы определить иррациональное число, представленное\\
	бесконечной непрерывной дробью \textit{x}=$[3; 6, \overline{1, 4}]$, запишем сначала \textit{x}=[3; 6, \textit{y}], где\\
	$$\textit{y}=[\overline{1; 4}]=[1; 4, \textit{y}]$$\\
	Тогда\\ $$\textit{y}=1+\frac{1}{4+1/\textit{y}}=1+\frac{\textit{y}}{4\textit{y}+1}=\frac{5\textit{y}+1}{4y+1},$$\\
	что приводит к квадратному уравнению \\
	$$4y^2-4y-1=0.$$\\
	Поскольку \textit{y}>0 и это уравнение имеет только один положительный корень, мы можем сделать вывод, что\\
	$$\textit{y}=\frac{1+\sqrt{2}}{2}.$$
	\newpage
	\renewcommand{\headrulewidth}{0pt}
	\lhead{\textbf{\thepage}}
	\setcounter{page}{318}
	\chead{\textbf{Числа Фибоначчи и непрерывные дроби}}
	\rhead{\textbf{Глава 13}}
	Из \textit{x}=[3; 6, \textit{y}] находим, что\\
	$$\textit{x}=3+\frac{1}{6+\frac{1}{\frac{1+\sqrt{2}}{2}}}=\frac{25+19\sqrt{2}}{8+6\sqrt{2}}=\frac{(25+19\sqrt{2})(8-6\sqrt{2})}{(8+6\sqrt{2})(8-6\sqrt{2})}=\frac{14-\sqrt{2}}{4};$$\\
	таким образом, $[3; 6, \overline{1, 4}]=\frac{14-\sqrt{2}}{4}.$\\\\
	\indent{Наша предыдущая теорема показывает, что каждая бесконечная\\
	непрерывная дробь представляет уникальное иррациональное число. Теперь же установим, наоборот, что любое иррациональное число может быть\\
	разложено в бесконечную непрерывную дробь $[a_0, a_1, a_2, ...]$, которая сходится к значению \textit{$x_0$}. Последовательность целых чисел $a_0, a_1, a_2, ...$ определяется следующим образом: используя скобочную функцию, мы сначала запишем}\\
	$$\textit{$x_1$}=\frac{1}{\textit{$x_0$}-[\textit{$x_0$}]}, \textit{$x_2$}=\frac{1}{\textit{$x_1$}-[\textit{$x_1$}]}, \textit{$x_3$}=\frac{1}{\textit{$x_2$}-[\textit{$x_2$}]}, ...$$\\
	а затем приравняем\\
	$$a_0=[x_0], a_1=[x_1], a_2=[x_2], a_3=[x_3], ... .$$\\
	В общем, $a_k$ получаются индуктивно из\\
	$$a_k=[x_k], x_{k+1}=\frac{1}{x_k-a_k}, (k\geq0).$$\\
	Очевидно, что $x_{k+1}$ $\textemdash$ иррациональное число всегда, когда $х_k$ иррационально; и поскольку мы ограничиваемся случаем, когда $x_0$ иррациональное, все $x_k$ иррациональны по индукции. Таким образом,\\
	$$0<x_k-a_k=x_k-[x_k]<1$$\\
	и мы видим, что\\
	$$x_{k+1}=\frac{1}{x_k-a_k}>1,$$\\
	так что целое число $a_k=[x_{k+1}]\geq1$ для всех $k\geq$0. Поэтому данный процесс приводит к бесконечной последовательности целых чисел $a_0, a_1, a_2, ...$, которые все, кроме, быть может,  $a_0$, положительные.

\end{document}